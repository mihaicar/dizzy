\vspace*{2.5cm}
    \LARGE
    \textbf{Abstract}

\normalsize
 \vspace{3cm}

Distributed ledger technology is enabling institutions to optimise their business flows by automating lengthy processes and reducing the effort of maintaining different financial accounts synchronised across all their contributors. A newly implemented platform based on this is Corda, developed specifically for the banking industry. This project focuses on implementing stock trading on Corda and evaluating whether the system is a feasible solution for commercial deployment. In order to do this, we had to ensure its performance, security and regulation controls are up to the standards required by the industry. Since data is shared on a need-to-know basis, not only are security concerns mitigated, but the scalability of the system is also improved compared to other solutions. Performance-wise, the system is expected to yield more than 80,000 transactions per second which is an appropriate value for the industry. From the point of view of regulation, we explored auditing facilities and evaluated the ways in which the system complies with the current financial laws. 

\vspace{0.5cm}
Besides theoretical analysis of features, we tested the privacy and scalability of the system by implementing two applications for a group of banks distributed over different machines: a virtual marketplace with a graphical interface and one where trades are randomly triggered between participants. These demonstrate the usability of the system, while showcasing the data-sharing principles in practice. Introducing new entities into the marketplace has also been achieved, without adding any latencies to the system. 